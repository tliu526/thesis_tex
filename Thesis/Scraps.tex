Possible Stencils include considering all cells that share an edge with the center cell the neighborhood or considering all cells that share a vertex with the center cell the neighborhood. These Stencils can be thought of as generalized von Neumman and Moore neighborhoods, respectively.

\begin{itemize}
\item lifetime to stability/ash density against lifetime replication
\item Long Single Penrose Life Run
\item Comparison between Kite/Darts and Rhomb Penrose Grids
\item Neighborhood analysis

\item Grid Size Comparison

\item Subregion Lifetime analysis
\end{itemize}


 
        \begin{subfigure}[t]{0.2\textwidth}
    \centering
    \includegraphics[width=\textwidth]{ch4_figs/crh_long/crh_long_\i}
    \end{subfigure} 
    ~
    \begin{subfigure}[t]{0.2\textwidth}
    \centering
    \includegraphics[width=\textwidth]{ch4_figs/crh_long/crh_long_\i}
    \end{subfigure}
    ~
    \begin{subfigure}[t]{0.2\textwidth}
    \centering
    \includegraphics[width=\textwidth]{ch4_figs/crh_long/crh_long_\i}
    \end{subfigure}
    ~
    \begin{subfigure}[t]{0.2\textwidth}
    \centering
    \includegraphics[width=\textwidth]{ch4_figs/crh_long/crh_long_\i}
    \end{subfigure}


\multido{\i=0+1}{66}{%
\subcaptionbox{}{\includegraphics[width=0.2\columnwidth]{ch4_figs/crh_long/crh_long_0}}&
\subcaptionbox{}{\includegraphics[width=0.2\columnwidth]{ch4_figs/crh_long/crh_long_1}}&
\subcaptionbox{}{\includegraphics[width=0.2\columnwidth]{ch4_figs/crh_long/crh_long_2}}&
\subcaptionbox{}{\includegraphics[width=0.2\columnwidth]{ch4_figs/crh_long/crh_long_3}}\\
}




\iffalse
\multido{\i=0+1}{66}{
    \subcaptionbox{\i}{\includegraphics[width=0.2\columnwidth]{ch4_figs/crh_long/crh_long_\i}}&
    \subcaptionbox{\i}{\includegraphics[width=0.2\columnwidth]{ch4_figs/crh_long/crh_long_\i}}&
    \subcaptionbox{\i}{\includegraphics[width=0.2\columnwidth]{ch4_figs/crh_long/crh_long_\i}}&
    \subcaptionbox{\i}{\includegraphics[width=0.2\columnwidth]{ch4_figs/crh_long/crh_long_\i}}\\
}
\fi

\iffalse
\begin{python}

for i in range(0,5):

    print("\subcaptionbox{}{\includegraphics[width=0.2\columnwidth]{ch4_figs/crh_long/crh_long_0}}&")
    print("\subcaptionbox{}{\includegraphics[width=0.2\columnwidth]{ch4_figs/crh_long/crh_long_0}}&")   
    print("\subcaptionbox{}{\includegraphics[width=0.2\columnwidth]{ch4_figs/crh_long/crh_long_0}}&")
    print("\subcaptionbox{}{\includegraphics[width=0.2\columnwidth]{ch4_figs/crh_long/crh_long_0}}\\")
\end{python}


\makeatletter
\newwrite\Code@out

\newcommand\python{\obeylines\expandafter\pythonArg\noexpand}

\newcommand\pythonArg[1][tmp.py.in]{%
    \gdef\FNameIn{#1}
    \gdef\FNameOut{tmp.py.out}
    \begingroup
        \@bsphack%
        \immediate\openout\Code@out\FNameIn%
        \let\do\@makeother\dospecials%
        \catcode`\^^M\active%
        \def\verbatim@processline{%
            \immediate\write\Code@out{\the\verbatim@line}}%
        \verbatim@start}

\def\endpython{%
        \immediate\closeout\Code@out\@esphack
    \endgroup

     %Execute python script. Python directory must be in PATH.
     \immediate\write18{python \FNameIn > \FNameOut}
     \input{\FNameOut}
}

\makeatother
